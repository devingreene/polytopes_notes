\documentclass[11pt]{article}

\usepackage{amsmath}
\usepackage{amsfonts}
\usepackage{amsthm}
\usepackage{xcolor}
\usepackage{mathtools}
\usepackage{enumerate}
\usepackage{listings}
\usepackage[affil-sl]{authblk}

%\numberwithin{equation}{section}

\theoremstyle{plain} \newtheorem{thm}{Theorem}%[section]
\theoremstyle{plain} \newtheorem{define}{Definition}%[section]
\theoremstyle{plain} \newtheorem{example}{Example}%[section]
\theoremstyle{plain} \newtheorem{remark}{Remark}%[section]

\newcommand{\reals}{\mathbb{R}}
\newcommand{\realn}{\mathbb{R}^n}
\newcommand{\mixed}{mixed-sign}

\begin{document}
\begin{define}
    A subspace $V$ of $\realn$ is called \mixed{} if every non-zero
    vector in $V$ has both negative and positive components.
\end{define}

\begin{thm}
    Let $V \subset \realn$ be a \mixed{} subspace.  Then there exists a
    vector $\bf t$ in $\realn$ with all positive components such that
    $\bf t$ is orthogonal to $V$.
\end{thm}

\begin{proof}
    Proof by induction on $n$.  $\reals^1$ has no \mixed{} subspace.
    Any \mixed{} subspace of $\reals^2$ has the form $\reals
    \langle a, -b\rangle$ where $a,b > 0$.  So ${\bf t} = \langle b,
    a\rangle$ satisfies the condition.

    Express a basis of $V$ as a matrix of row vectors:

    \begin{equation*}
    \begin{pmatrix}
    v_{11}&v_{12}&\cdots&v_{1n}\cr
    \vdots&\vdots&\ddots&\vdots\cr
    v_{m1}&v_{m2}&\cdots&v_{mn}\cr
    \end{pmatrix}
    \end{equation*}

    If the first column is all zeros, then by induction there exists
    ${\bf t}_0 \in \reals{}^{n-1}$ orthogonal to all the row vectors of
    the submatrix consisting of the right $n-1$ colunns.  Extend
    ${\bf t}_0$ to a satisfying vector $\bf t$ by prepending an arbitrary
    positive number.

    If the first column is not zero, use elimination to derive a matrix
    of the form

    \begin{equation*}
    \begin{pmatrix}
    1&v_{12}&\cdots&v_{1n}\cr
    0&v_{22}&\cdots&v_{2n}\cr
    \vdots&\vdots&\ddots&\vdots\cr
    0&v_{m2}&\cdots&v_{mn}\cr
    \end{pmatrix}
    \end{equation*}

    Using standard orthogonalization moves, we can assume that top row
    is orthogonal to the remaining rows.

    \medskip
    \noindent Two cases:

    \noindent Case 1:

    The subspace of $\reals^{n-1}$ spanned by the rows of the
    matrix consisting of the right $n-1$ columns is a \mixed{} subspace.

    \medskip

    By induction, there exists a vector ${\bf t_p} = \langle 0,t'_2,\cdots
    t'_n\rangle$, with\break$t'_2, t'_3, \ldots, t'_n > 0$ which is orthogonal to the rows of the matrix.

    Perturb ${\bf t_p}$ slightly by \epsilon$\langle0, v_{12}, \ldots
    v_{1n}\rangle$ with $\epsilon > 0$ so that the resulting vector

    $${\bf \tilde{t}} = \langle 0, t_2, \ldots, t_n\rangle =
    {\bf t_0} - \epsilon \langle 0,v_{12},\ldots,v_{1n}\rangle$$
    still has components $t_2,\ldots, t_n > 0$.

    Clearly ${\bf \tilde{t}}$ is orthogonal to the lower $m-1$
    rows.  Its inner product with the top row is

    $$-\epsilon \|\langle v_{12},\ldots,v_{1n}\rangle\|^2$$

    This is a negative number since $\langle v_{12},v_{13},\ldots,v_{1n}\rangle$
    has \mixed{}.  The
    vector
    $${\bf t} = \langle \epsilon \|\langle v_{12}, \ldots
    v_{1n}\rangle\|^2, t_2, \ldots, t_n\rangle$$
    satisfies the desired condition.
    
    \medskip

    \noindent Case 2:

    The subspace referred to in Case 1 is not a
    \mixed{} subspace.

    In this case, it must be possible to find a nonzero linear
    combination $\textbf{x}$ of the rows of the matrix where the last
    $n-1$ components of $\textbf{x}$ are $\le 0$ and are not all
    nonzero. Since the lower $m-1$ rows are in $V$, they span a \mixed{}
    subspace.  Thus the coefficient of the top row in the expression for
    $\textbf{x}$ must be nonzero. We can then swap the top row of the
    matrix with $\textbf{x}$ without changing the row space. Since
    $\textbf{x}$ is in the \mixed{} row-space under consideration, the
    first component must be $>0$, so we assume that it is $1$.

    By induction, there exists a vector ${\bf t_p} =
    \langle0,t_2,\ldots,t_n\rangle$ such that\break$t_2,\ldots,t_n > 0$ and
    which is orthogonal to the last $m-1$ rows.  The inner product of
    ${\bf t_p}$ with the first row must be a negative number
    $\alpha$.  Take ${\bf t} = \langle -\alpha,t_2,\ldots,t_n\rangle$
    This vector is orthogonal to all the rows of the matrix, and thus to
    $V$.


    
\end{proof}
\end{document}
% vim: tw=72
